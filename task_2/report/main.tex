\documentclass[a4paper,12pt]{article}
\usepackage[utf8x]{inputenc}
\usepackage[T2A]{fontenc}
\usepackage[russian,english]{babel}
\usepackage{amsmath}
\usepackage{cmap}
\usepackage{booktabs}
\usepackage{caption}
\usepackage{enumitem}
\usepackage{listings}
\usepackage{xcolor}
\usepackage{setspace}
\usepackage[left=3cm, right=1.5cm, top=2cm, bottom=2cm]{geometry}
\renewcommand{\labelenumii}{\arabic{enumi}.\arabic{enumii}.}
\lstset{
    language=C++,
    basicstyle=\small\ttfamily,
    keywordstyle=\color{blue},
    commentstyle=\color{green!40!black},
    stringstyle=\color{purple},
    numbers=left,
    numberstyle=\tiny,
    numbersep=5pt,
    breaklines=true,
    frame=single,
    backgroundcolor=\color{gray!10},
    rulecolor=\color{black!30},
    showstringspaces=false,
    extendedchars=\true, % Включение расширенных символов, включая русский текст
}
\begin{document}

\begin{center}
\hfill \break
\textbf{\large{Министерство науки и высшего образования Российской Федерации\\
Федеральное государственное автономное образовательное\\
учреждение высшего образования}}
\\
\large{\textbf{«КАЗАНСКИЙ (ПРИВОЛЖСКИЙ) ФЕДЕРАЛЬНЫЙ УНИВЕРСИТЕТ»}}\\
\hfill \break
\large{ИНСТИТУТ ВЫЧИСЛИТЕЛЬНОЙ МАТЕМАТИКИ\\ И ИНФОРМАЦИОННЫХ ТЕХНОЛОГИЙ}\\
 \hfill \break
\large{Кафедра прикладной математики и искусственого интеллекта}\\
\hfill\break
\hfill \break
\large{Направление подготовки: 01.03.04 – Прикладная математика}\\
\hfill \break
\hfill \break
\textbf{\large{ОТЧЁТ}}\\
\large{По дисциплине <<Численные методы>>}\\
\large{на тему:}\\
\large{<<Вычисление интеграла с помощью квадратурных формул>>}\\
\hfill \break
\hfill \break
\end{center}

\hfill \break
\begin{flushright}
			
    \large{Выполнил:}

    \large{студент группы 09-221}
    
    \large{Саитов М.А.}
    
    \large{Проверил:}
    
    \large{ассистент Глазырина О.В.}
    
\end{flushright}
\vfill
\begin{center} \large{Казань, 2024 год} \end{center}
\thispagestyle{empty}
 
\newpage
\begin{center}
\renewcommand{\contentsname}{Содержание}
\fontsize{14}{1.15}\selectfont
\mdseries\selectfont{\tableofcontents}
\newpage
\end{center}
\setlength{\parindent}{1.25cm}
\newpage
\selectfont\onehalfspacing{
\section{Постановка задачи}
\hspace{1.25cm}Необходимо изучить и сравнить различные способы приближённого вычисления функции ошибок
%  \begin{equation}
%     \text{erf}(x) =\displaystyle\frac{2}{\sqrt{\pi}} \displaystyle\sum\limits_{n=1}^{\infty}(-1)^n\displaystyle\frac{x^{2n+1}}{n!(2n+1)}
% \end{equation}
\begin{equation}
    \text{erf}(x) =\displaystyle\frac{2}{\sqrt{\pi}} \displaystyle\int\limits_{0}^{x} e^{-t^2} dt
\end{equation}
\begin{enumerate}
    \item Протабулировать erf(x) на отрезке [a,b] с шагом h и точностью $ \varepsilon$, основываясь на ряде Маклорена,
     предварительно вычислив его. Получив таким образом таблицу из 11 точек вида:
    \begin{tabbing}
        $x_0$ \= $x_1$ \= $x_2$ \= \dots\\
        $f_0$ \> $f_1$ \> $f_2$ \> \dots\\
    \end{tabbing}
    $f_i = \text{erf}(x_i), \quad x_i = a + i  h, \quad i = 0,\dots,n.$
    \item {Вычислить erf(x) при помощи пяти составных квадратурных формул при\\
     $h=(x_{i+1} - x_i)$, 
        \begin{equation}   
           \text{g}(x) =\displaystyle\frac{2}{\sqrt{\pi}} e^{-x^2}
        \end{equation}
        \begin{enumerate}
            \item {
                Формула правых прямоугольников:
                \begin{equation}
                    J_N(x) = \displaystyle\sum_{i=1}^{n}h g(x_i)
                \end{equation}
                }
            \item {
                Формула центральных прямоугольников:
                \begin{equation}
                    J_N(x) = \displaystyle\sum_{i=1}^{n}h g\left(\displaystyle\frac{x_i + x_{i+1}}{2}\right) 
                \end{equation}
                }
            \item {
                Формула трапеции:
                \begin{equation}
                    J_N(x) = \displaystyle\sum_{i=1}^{n}h \displaystyle\frac{g(x_i)+g(x_{i+1})}{2}
                \end{equation}
                }
            \item {
                Формула Симпсона:
                \begin{equation}
                    J_N(x) = \displaystyle\sum_{i=1}^{n}\displaystyle\frac{h}{6} \left[g(x_i)+4g\left(
                        \displaystyle\frac{x_i + x_{i+1}}{2}
                    \right) + g(x_{i+1}) \right]
                \end{equation}
                }
            \item {
                Формула Гаусса:
                \begin{equation}
                    J_N(x) = \displaystyle\sum_{i=1}^{n}\displaystyle\frac{h}{2} \left[g\left(
                        x_i + \displaystyle\frac{h}{2}\left(1 - \displaystyle\frac{1}{\sqrt{3}} \right)
                    \right) +
                    g\left(
                        x_i + \displaystyle\frac{h}{2}\left(1 + \displaystyle\frac{1}{\sqrt{3}} \right)
                    \right)
                    \right]
                \end{equation}
                }
        \end{enumerate}
    }
\end{enumerate}

Вычисления проводятся от начала интегрирования до каждой из 11 точек, увели\-чивая количество разбиений между точками в 2 раза до тех пор, пока погрешность больше $\varepsilon$.
\newpage
\section{Ход работы}
\hspace*{1.25cm}Для того чтобы найти значение функции в точке, необходимо протабулировать искомый интеграл на отрезке [a, b] с шагом $h = 0.2$ и точностью $\varepsilon$. Для этого:
\begin{enumerate}
    \item Найдём разбиение подинтегральной функции $e^{-t^2}$ в ряд Маклорена, подставив в стандартное разбиение функции $e^x$ в ряд Маклорена $x = -t^2$:
    \begin{equation}
        e^x = \sum_{n = 0}^{\infty}\frac{x^n}{n!} \implies e^{-t^2} = \sum_{n = 0}^{\infty}\frac{(-1)^n t^{2n}}{n!}
    \end{equation}
    \item Проинтегрируем полученное выражение на интеграле [0, x]:
    \begin{equation}
        \frac{2}{\sqrt{\pi}}\int\limits_{0}^{x}\sum_{n = 0}^{\infty}\frac{(-1)^n t^{2n}}{n!}\mathrm{d}t = \frac{2}{\sqrt{\pi}}\left.\sum_{n = 0}^{\infty}\frac{(-1)^n t^{2n+1}}{(2n-1)n!}\right|_{0}^{x} = \frac{2}{\sqrt{\pi}}\sum_{n = 0}^{\infty}\frac{(-1)^n x^{2n+1}}{(2n-1)n!}
    \end{equation}
    \item Выделим два общих члена $a_n, \; a_{n + 1}$ из полученного выражения и найдём\\ $q_n = \dfrac{a_{n+1}}{a_n}$:
    \begin{equation}
        a_{n}=\frac{(-1)^{n}x^{2n+1}}{n!(2n+1)}, \quad a_{n+1}=\frac{(-1)^{n+1}x^{2n+3}}{(n+1)!(2n+3)}.
    \end{equation}
    \begin{equation}
        q_{n}=\frac{-x^{2}(2n+1)}{(n+1)(2n+3)}.
    \end{equation}
\end{enumerate}
Таким образом,
\begin{equation}
	a_{n}=a_{0}\prod_{n=0}^{n-1} q_{n}.
\end{equation}
Для каждой точки $x_i = a + i  h$ найдём значение $erf(x_i)$ и составим таблицу результатов (Таблица 1).
\begin{table}[h]
    \centering
    \begin{tabular}{|c|c|}
        \hline
        $x_i$ & $erf(x_i)$\\
        \hline
        0,0 & 0,0000000000\\
        \hline
        0,2 & 0,2227025926\\
        \hline
        0,4 & 0,4283923805\\
        \hline
        0,6 & 0,6038561463\\
        \hline
        0,8 & 0,7421009541\\
        \hline
        1,0 & 0,8427006602\\
        \hline
        1,2 & 0,9103140831\\
        \hline
        1,4 & 0,9522852302\\
        \hline
        1,6 & 0,9763484001\\
        \hline
        1,8 & 0,9890906215\\
        \hline
        2,0 & 0,9953226447\\
        \hline
    \end{tabular}
    \caption*{\small{Таблица 1 - точки $x_i$ и значения разложения в ряд Маклорена функции $erf(x_i)$}}
\end{table}
\newpage
После нахождения значений разложения в ряд Маклорена в точках, вычислим значение erf(x) при помощи 5 составных квадратурных формул. Для каждой формулы составим свою таблицу. В таблицах будут находится значения точки, для которой производились расчёты, значение разбиения в ряд Маклорена в точке, значение найден\-ного с помощью формулы интеграла в точке, модуль разницы между значениями найде\-нного интеграла и разбиения, количества разбиений, которые пришлось совершить для нахожде\-ния значения интеграла с нужной точностью.
\begin{enumerate}[label = \arabic*.]
    \item {Левые прямоугольники:
        \begin{table}[h]
        \centering
        \begin{tabular}{|c|c|c|c|c|}
            \hline
            $x_i$ & $J_0(x_i)$ & $J_(x_i)$ & $\left|J_0(x_i) - J_N(x_i)\right|$ & $N$\\
            \hline
            0.2 &  0.222703 &  0.222707 & 4.32204e-06 & 1024\\
            \hline
            0.4 &  0.428392 &  0.428425 & 3.25766e-05 & 1024\\
            \hline
            0.6 &  0.603856 &  0.603956 &  9.9908e-05 & 1024\\
            \hline
            0.8 &  0.742101 &  0.742309 & 0.000208343 & 1024\\
            \hline
            1.0 &  0.842701 &  0.843049 & 0.000348225 & 1024\\
            \hline
            1.2 &  0.910314 &  0.910818 & 0.000504366 & 1024\\
            \hline
            1.4 &  0.952285 &  0.952948 &  0.00066258 & 1024\\
            \hline
            1.6 &  0.976348 &  0.977162 &  0.00081329 & 1024\\
            \hline
            1.8 &  0.989091 &  0.990043 & 0.000952791 & 1024\\
            \hline
            2.0 &  0.995322 &  0.996404 &  0.00108161 & 1024\\
            \hline
        \end{tabular}
        \caption*{\small{Таблица 2 - таблица значений для формулы Левых прямоугольников}}
        \end{table}
    }
    \item {Правые прямоугольники:
        \begin{table}[h]
          \centering
          \begin{tabular}{|c|c|c|c|c|}
            \hline
            $x_i$ & $J_0(x_i)$ & $J_(x_i)$ & $\left|J_0(x_i) - J_N(x_i)\right|$ & $N$\\
            \hline
            0.2 &  0.222703 &  0.222698 & 4.31945e-06 & 1024\\
            \hline
            0.4 &  0.428392 &   0.42836 & 3.25944e-05 & 1024\\
            \hline
            0.6 &  0.603856 &  0.603756 & 9.99763e-05 & 1024\\
            \hline
            0.8 &  0.742101 &  0.741893 & 0.000208371 & 1024\\
            \hline
            1.0 &  0.842701 &  0.842352 &  0.00034833 & 1024\\
            \hline
            1.2 &  0.910314 &  0.909809 & 0.000504659 & 1024\\
            \hline
            1.4 &  0.952285 &  0.951622 & 0.000662823 & 1024\\
            \hline
            1.6 &  0.976348 &  0.975535 & 0.000813508 & 1024\\
            \hline
            1.8 &  0.989091 &  0.988138 & 0.000953008 & 1024\\
            \hline
            2.0 &  0.995322 &   0.99424 &  0.00108189 & 1024\\
            \hline
          \end{tabular}
          \caption*{\small{Таблица 3 - таблица значений для формулы Правых прямоугольников}}
        \end{table}
        \newpage
    }
    \item {Центральные прямоугольники:
        \begin{table}[h]
          \centering
          \begin{tabular}{|c|c|c|c|c|}
            \hline
            $x_i$ & $J_0(x_i)$ & $J_(x_i)$ & $\left|J_0(x_i) - J_N(x_i)\right|$ & $N$\\
            \hline
            0.2 &  0.222703 &  0.222703 & 1.80774e-07 & 64\\
            \hline
            0.4 &  0.428392 &  0.428393 & 3.14933e-07 & 128\\
            \hline
            0.6 &  0.603856 &  0.603856 & 2.09718e-07 & 256\\
            \hline
            0.8 &  0.742101 &  0.742101 & 1.31198e-07 & 512\\
            \hline
            1.0 &  0.842701 &  0.842701 & 1.45324e-07 & 512\\
            \hline
            1.2 &  0.910314 &  0.910314 & 7.42506e-08 & 512\\
            \hline
            1.4 &  0.952285 &  0.952285 & 8.73705e-08 & 512\\
            \hline
            1.6 &  0.976348 &  0.976348 & 6.19822e-08 & 512\\
            \hline
            1.8 &  0.989091 &  0.989091 & 2.63449e-07 & 256\\
            \hline
            2.0 &  0.995322 &  0.995322 & 9.90524e-08 & 256\\
            \hline
          \end{tabular}
          \caption*{\small{Таблица 4 - таблица значений для формулы Центральных прямоугольников}}
        \end{table}
    }
    \item {Формула трапеций:
        \begin{table}[h]
          \centering
          \begin{tabular}{|c|c|c|c|c|}
            \hline
            $x_i$ & $J_0(x_i)$ & $J_(x_i)$ & $\left|J_0(x_i) - J_N(x_i)\right|$ & $N$\\
            \hline
            0.2 &  0.222703 &  0.222703 & 8.39077e-08 & 128\\
            \hline
            0.4 &  0.428392 &  0.428392 & 1.55199e-07 & 256\\
            \hline
            0.6 &  0.603856 &  0.603856 & 1.16408e-07 & 512\\
            \hline
            0.8 &  0.742101 &  0.742101 & 1.59468e-07 & 512\\
            \hline
            1.0 &  0.842701 &  0.842701 & 2.50953e-07 & 512\\
            \hline
            1.2 &  0.910314 &  0.910314 & 3.70479e-07 & 512\\
            \hline
            1.4 &  0.952285 &  0.952285 &  3.2974e-07 & 512\\
            \hline
            1.6 &  0.976348 &  0.976348 & 2.80325e-07 & 512\\
            \hline
            1.8 &  0.989091 &   0.98909 & 2.31385e-07 & 512\\
            \hline
            2.0 &  0.995322 &  0.995322 & 2.17437e-07 & 512\\
            \hline
          \end{tabular}
          \caption*{\small{Таблица 5 - таблица значений для формулы Трапеций}}
        \end{table}
        \newpage
    }
    \item Формула Симпсона
        \begin{enumerate}
            \item {Вывод формулы Симпсона через интегральный полином Лагранжа:\\
            Формула для полинома Лагранжа:
            \begin{equation}
                L_n(x) = \sum_{i=0}^{n}f(x_i)\prod_{i \ne j, j = 0}^{n}\frac{x - x_j}{x_i - x_j}
            \end{equation}
            По трём узлам $(x_1 = a, x_2 = \dfrac{a+b}{2}, x_3 = b):\\
            L_2 = f(a)\left(\dfrac{x - \dfrac{a+b}{2}}{a - \dfrac{a+b}{2}}\right)\left(\dfrac{x - b}{a - b}\right)+
               f\left(\dfrac{a +b}{2}\right)\left(\dfrac{x-a}{\dfrac{a+b}{2} - a}\right)\left(\dfrac{x-b}{\dfrac{a+b}{2} - b}\right)+\\
               f(b)\left(\dfrac{x - \dfrac{a+b}{2}}{b - \dfrac{a+b}{2}}\right)\left(\dfrac{x - b}{b - a}\right).$\\
            \hfill\break
            Проинтегрируем выражение по интервалу [a,b]:
            \begin{equation}
                \int\limits_{a}^{b}L_2(x)\mathrm{d}x = f(a)c_1 + f\left(\frac{a+b}{2}\right)c_2 + f(b)c_3
            \end{equation}
            где $c_1 = \dfrac{b-a}{6}, c_2 = \dfrac{2}{3}(b - a), c_3 = \dfrac{b-a}{6}.$\\
            \hfill\break
            Тогда:
            \begin{equation}
                \int\limits_{a}^{b}L_2(x)\mathrm{d}x = \frac{b-a}{6}\left(f(a) + 4f\left(\frac{a+b}{2}\right)+f(b)\right)
            \end{equation}
            }
            \item {Значения полученные для формулы Симпсона:
            \begin{table}[h]
            \centering
            \begin{tabular}{|c|c|c|c|c|}
                \hline
                $x_i$ & $J_0(x_i)$ & $J_(x_i)$ & $\left|J_0(x_i) - J_N(x_i)\right|$ & $N$\\
                \hline
                0.2 &  0.222703 &  0.222703 & 1.20186e-08 & 4\\
                \hline
                0.4 &  0.428392 &  0.428392 & 1.23149e-08 & 8\\
                \hline
                0.6 &  0.603856 &  0.603856 & 4.87389e-08 & 8\\
                \hline
                0.8 &  0.742101 &  0.742101 & 4.40568e-08 & 16\\
                \hline
                1.0 &  0.842701 &  0.842701 & 3.49238e-08 & 16\\
                \hline
                1.2 &  0.910314 &  0.910314 & 6.97319e-08 & 8\\
                \hline
                1.4 &  0.952285 &  0.952285 &  7.4049e-08 & 16\\
                \hline
                1.6 &  0.976348 &  0.976348 & 9.72274e-08 & 16\\
                \hline
                1.8 &  0.989091 &   0.98909 & 1.38559e-07 & 16\\
                \hline
                2.0 &  0.995322 &  0.995322 & 1.00857e-07 & 32\\
                \hline
            \end{tabular}
            \caption*{\small{Таблица 6 - таблица значений для формулы Симпсона}}
            \end{table}
            }
        \end{enumerate}
        \newpage

    \item {Формула Гаусса:
        \begin{table}[h]
          \centering
          \begin{tabular}{|c|c|c|c|c|}
            \hline
            $x_i$ & $J_0(x_i)$ & $J_(x_i)$ & $\left|J_0(x_i) - J_N(x_i)\right|$ & $N$\\
            \hline
            0.2 &  0.222703 &  0.222703 & 9.76378e-11 & 4\\
            \hline
            0.4 &  0.428392 &  0.428392 & 9.54019e-09 & 8\\
            \hline
            0.6 &  0.603856 &  0.603856 & 3.76879e-08 & 8\\
            \hline
            0.8 &  0.742101 &  0.742101 & 2.71688e-08 & 16\\
            \hline
            1 &  0.842701 &  0.842701 & 5.74231e-09 & 16\\
            \hline
            1.2 &  0.910314 &  0.910314 & 9.95342e-08 & 8\\
            \hline
            1.4 &  0.952285 &  0.952285 & 4.75374e-08 & 16\\
            \hline
            1.6 &  0.976348 &  0.976348 & 9.31051e-09 & 16\\
            \hline
            1.8 &  0.989091 &  0.989091 & 2.59625e-08 & 16\\
            \hline
            2 &  0.995322 &  0.995322 & 4.66851e-08 & 16\\
            \hline
          \end{tabular}
          \caption*{\small{Таблица 7 - таблица значений для формулы Гаусса}}
        \end{table}
    }
\end{enumerate}
\section{Выводы}
\hspace{1.25cm}Проделав все вычисления, можно сделать выводы, что более комплексные методы вычисления интеграла, как формула Гаусса и Симпсона, показыают наилучшие результа\-ты за меньшее количество разбиений. В это же время худшие результаты вычисления показыают методы правых прямоугольников и метод трапеций, приводя к довольно большому значению ошибки.}
\newpage
\section{Листинг программы}
\lstinputlisting[language=C++]{../src/Function.h}
\lstinputlisting[language=C++]{../src/Function.cpp}
\lstinputlisting[language=C++]{../src/main.cpp}
\end{document}