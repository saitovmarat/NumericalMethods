\documentclass[a4paper,12pt]{article}
\usepackage[utf8x]{inputenc}
\usepackage[T2A]{fontenc}
\usepackage[russian,english]{babel}
\usepackage{amsmath}
\usepackage{cmap}
\usepackage{booktabs}
\usepackage{caption}
\usepackage{enumitem}
\usepackage{listings}
\usepackage{xcolor}
\usepackage{setspace}
\usepackage{graphicx}
\usepackage[left=2cm, right=1.5cm, top=2cm, bottom=2cm]{geometry}
\renewcommand{\labelenumii}{\arabic{enumi}.\arabic{enumii}.}
\lstset{
    language=C++,
    basicstyle=\small\ttfamily,
    keywordstyle=\color{blue},
    commentstyle=\color{green!40!black},
    stringstyle=\color{purple},
    numbers=left,
    numberstyle=\tiny,
    numbersep=5pt,
    breaklines=true,
    frame=single,
    backgroundcolor=\color{gray!10},
    rulecolor=\color{black!30},
    showstringspaces=false,
    extendedchars=\true, % Включение расширенных символов, включая русский текст
}
\begin{document}

\begin{center}
\hfill \break
\textbf{\large{Министерство науки и высшего образования Российской Федерации\\
Федеральное государственное автономное образовательное\\
учреждение высшего образования}}
\\
\large{\textbf{«КАЗАНСКИЙ (ПРИВОЛЖСКИЙ) ФЕДЕРАЛЬНЫЙ УНИВЕРСИТЕТ»}}\\
\hfill \break
\large{ИНСТИТУТ ВЫЧИСЛИТЕЛЬНОЙ МАТЕМАТИКИ\\ И ИНФОРМАЦИОННЫХ ТЕХНОЛОГИЙ}\\
 \hfill \break
\large{Кафедра прикладной математики и искусственого интеллекта}\\
\hfill\break
\hfill \break
\large{Направление подготовки: 01.03.04 – Прикладная математика}\\
\hfill \break
\hfill \break
\textbf{\large{ОТЧЁТ}}\\
\large{По дисциплине <<Численные методы>>}\\
\large{на тему:}\\
\large{<<Система линейных алгебраических уравнений>>}\\
\hfill \break
\hfill \break
\end{center}

\hfill \break
\begin{flushright}
			
    \large{Выполнил:}
    
    \large{студент группы 09-221}
    
    \large{Саитов М.А.}
    
    \large{Проверил:}
    
    \large{ассистент Глазырина О.В.}
    
\end{flushright}
\vfill
\begin{center} \large{Казань, 2024 год} \end{center}
\thispagestyle{empty}
 

\newpage
\begin{center}
\renewcommand{\contentsname}{Содержание}
\fontsize{14}{1.15}\selectfont
\mdseries\selectfont{\tableofcontents}
\end{center}
\newpage

\setlength{\parindent}{1.25cm}
\newpage
\selectfont\onehalfspacing

\begin{center}
\section{Постановка задачи}
\end{center}
\hspace{1.25cm}Решить систему линейных алгебраических уравнений:
\begin{equation}\label{eq:main_sys}
    \begin{cases}
        &(a_1 + a_2 + h^2g_1)y_1 - a_2y_2 = f_1h^2,\\
        &\hspace{1cm}\dots \quad \dots \quad \dots \quad \dots\\
        &-a_i y_{i-1} + (a_i + a_{i+1} + h^2g_i)y_i - a_{i+1}y_{i+1} = f_i h^2,\\
        &\hspace{1cm}\dots \quad \dots \quad \dots \quad \dots\\
        &(a_{n-1} + a_{n} + h^2g_{n-1})y_{n-1} - a_{n-1}y_{n-2} = f_{n-1} h^2.\\
    \end{cases}
\end{equation}

Здесь $a_i = p(ih),\;g_i = q(ih),\;f_i = f(ih),\;f(x) = -(p(x)u'(x))' + q(x)u(x),\;\\h = 1/n,\;p,\;q,\;u\;~-$ заданные функции.

Данную систему решить методом прогонки и итерационными методами:
\begin{enumerate}[label = \arabic*.]
    \item Якоби,
    \item верхней релаксации,
    \item наискорейшего спуска.
\end{enumerate}

Во всех итерационных методах вычисления продолжать до выполнения условия:
\begin{equation*}
    \max_{1 \le i \le n - 1} \left|r_{i}^{k}\right| \le \varepsilon,
\end{equation*}
$r\;~-$ вектор невязки, $\varepsilon\;~-$ заданное число.

\textbf{Исходные данные:} $n_1 = 10,\; n_2 = 20,\;\varepsilon = h^3,\;u(x) = x^{\alpha}(1-x)^{\beta},\;\\p(x) = 1 + x^{\gamma},\;g(x) = x + 1,\;\alpha = 4,\;\beta = 1,\;\gamma = 1.$

Для сравнения результатов вычисления составим таблицы и подведём выводы.
\newpage

\begin{center}
\section{Ход работы}
\end{center}
\subsection{Метод прогонки}

\hspace{1cm} Метод прогонки является частным случаем метода Гаусса и применяется
для решения систем линейных уравнений с трёхдиагональной матрицей. 
Метод прогонки состоит из двух этапов: прямой ход (определение прогоночных коэффициентов), обратный ход (вычисление неизвестных $x_k$).

Основными его преимуществами являются простота в реализации и то, что он максимально основан на структуре исходной системы.

Недостатком метода является то, что с каждой итерацией накапливается ошибка округления.

Запишем систему (1) в следующем виде:
\begin{equation}
  \begin{cases}
      &-b_1 y_1 + c_1 y_2 = f_1,\\
      &a_2 y_1 - b_2 y_2 + c_2 y_3 = f_2,\\
      &\dots \quad \dots \quad \dots \quad \dots,\\
      &a_i y_{i-1} - b_i y_i + c_i y_{i+1} = f_i,\\
      &\dots \quad \dots \quad \dots \quad \dots,\\
      &-a_{n-1} y_{n-1} + b_n y_n = f_n\\
  \end{cases}
\end{equation}

где a, b, c - значения, полученные при заполнении системы (1).

Разрешим первое уравнение системы (2) относительно $x_1$ и получим:

\begin{equation*}
  y_1 = \alpha_2 y_2 + \beta_2, \hspace{0.5cm} \alpha_2 = \frac{c_1}{b_1}, \hspace{0.5cm} \beta_2 = -\frac{f_1}{b_1}.
\end{equation*}

Из i-того уравнения системы (2) получим:

\begin{equation*}
  y_i = \alpha_{i+1} y_{i+1} + \beta_{i+1}, \hspace{0.5cm} i = \overline{1, n- 1}.
\end{equation*}

Таким образом, прямой ход будет заключаться в нахождении прогоночных коэффициентов:

\begin{align}
  \alpha_{i + 1} = \frac{c_i}{b_i - \alpha_i a_i}, \hspace{0.5cm} i = \overline{2, n - 1}, \hspace{0.5cm} \alpha_1 = \frac{c_0}{b_0}. \\
  \beta_{i + 1} = \frac{\beta_i a_i - f_i}{b_i - \alpha_i a_i}  \hspace{0.5cm} i = \overline{2, n - 1}, \hspace{0.5cm} \beta_1 = -\frac{f_0}{b_0}. 
\end{align}

Обратный ход будет заключаться в вычислении формул для нахождения неизветсных:

\begin{equation}
  \begin{cases}
    &y_i = \alpha_{i+1} y_{i+1} + \beta{i+1}, i = \overline{n - 1, 0},\\
    &y_n = \beta_{n+1}.\\ 
  \end{cases}
\end{equation}
\newpage

Формулы (3-5) описывают метод Гаусса, то есть метод прогонки.

Таким образом, после проделанных вычислений составим таблицу для n = 10 и n = 20 
с получившимися значениями, в которой первый столбец это номер итерации умноженный на число узлов разбиения.
Второй столбец - решение метода. Третий - значения функции в точке. В четвертом столбце находится значения погрешности.
\begin{enumerate}[label = \arabic*.]
    \item {Метод прогонки для n = 10:
        \begin{table}[h]
        \centering
        \begin{tabular}{|c|c|c|c|}
            \hline
            $ih$ & $y_i$ & $u(ih)$ & $\left|y_i-u(ih)\right|$\\ \hline
            0.1 & -0.000695158 &        9e-05 &  0.000785158\\ \hline
            0.2 &  -0.00039375 &      0.00128 &   0.00167375\\ \hline
            0.3 &   0.00300748 &      0.00567 &   0.00266252\\ \hline
            0.4 &    0.0116375 &      0.01536 &   0.00372252\\ \hline
            0.5 &    0.0264841 &      0.03125 &   0.00476594\\ \hline
            0.6 &    0.0462174 &      0.05184 &   0.00562257\\ \hline
            0.7 &    0.0660077 &      0.07203 &   0.00602232\\ \hline
            0.8 &    0.0763388 &      0.08192 &   0.00558118\\ \hline
            0.9 &    0.0618207 &      0.06561 &   0.00378929\\ \hline
        \end{tabular}
        \caption*{\small{Таблица 1 - таблица значений для формул метода прогонки при n = 10}}
        \end{table}
    }
    \item {Метод прогонки для n = 20:
      \begin{table}[h]
      \centering
      \begin{tabular}{|c|c|c|c|}
          \hline
          $ih$ & $y_i$ & $u(ih)$ & $\left|y_i-u(ih)\right|$\\ \hline
          0.05 & -7.79134e-05 &   5.9375e-06 &  8.38509e-05\\ \hline
          0.1 &  -8.5795e-05 &        9e-05 &  0.000175795\\ \hline
          0.15 &  0.000152985 &  0.000430313 &  0.000277328\\ \hline
          0.2 &  0.000889159 &      0.00128 &  0.000390841\\ \hline
          0.25 &   0.00241095 &   0.00292969 &   0.00051874\\ \hline
          0.3 &   0.00500727 &      0.00567 &  0.000662726\\ \hline
          0.35 &   0.00893085 &   0.00975406 &  0.000823217\\ \hline
          0.4 &    0.0143611 &      0.01536 &  0.000998868\\ \hline
          0.45 &    0.0213673 &    0.0225534 &   0.00118616\\ \hline
          0.5 &    0.0298709 &      0.03125 &   0.00137908\\ \hline
          0.55 &     0.039609 &    0.0411778 &    0.0015688\\ \hline
          0.6 &    0.0500966 &      0.05184 &   0.00174344\\ \hline
          0.65 &    0.0605893 &    0.0624772 &   0.00188784\\ \hline
          0.7 &    0.0700467 &      0.07203 &   0.00198335\\ \hline
          0.75 &    0.0770939 &    0.0791016 &   0.00200763\\ \hline
          0.8 &    0.0799855 &      0.08192 &    0.0019345\\ \hline
          0.85 &    0.0765672 &    0.0783009 &   0.00173375\\ \hline
          0.9 &     0.064239 &      0.06561 &     0.001371\\ \hline
          0.95 &    0.0399178 &    0.0407253 &  0.000807495\\ \hline
      \end{tabular}
      \caption*{\small{Таблица 2 - таблица значений для формул метода прогонки при n = 20}}
      \end{table}
    }
\end{enumerate}
\newpage



\subsection{Метод Якоби}  

\hspace{1cm} Для больших систем предпочтительнее оказываются итерационные методы. Основная идея этих методов состоит 
в построении последовательности векторов $x_k$, k = 1, 2, \dots, сходящихся к решению системы $Ax = b$.

За приближенное решение принимается вектор $x^k$ при достаточно большом $k$. При реализации итераицоннных методов,
обычно, достаточно уметь вычилсять вектор $Ax$ при любом заданном векторе $x$.

Будем считать, что все диагональные элементы матрицы из полной системы $Ac = b$ отличны от нуля, и перепишем
эту систему, разрешая каждое уравнение относительно переменной, стоящей на диагонали:

\begin{equation}
  x_i = \sum\limits_{j=1}^{i-1}\frac{a_{ij}}{a_{ii}} x_j - \sum\limits_{j = i+1}^{n}\frac{a_{ij}}{a_{ii}} x_j + \frac{b_i}{a_{ii}}, \hspace{0.5cm} i = \overline{1, n}.
\end{equation}

Выберем некоторое начальное приближение $x^0 = (x^0_1, x^0_2, \dots, x^0_n)^T$ и построим \\
последовательность векторов $x^1, x^2, \dots$ определяя вектор $x^{k+1}$ по уже найденному вектору $x^k$ при помощи соотношений:

\begin{equation}
  x^{k+1}_i = \sum\limits_{j=1}^{i-1}\frac{a_{ij}}{a_{ii}} x^k_j - \sum\limits_{j = i+1}^{n}\frac{a_{ij}}{a_{ii}} x^k_j + \frac{b_i}{a_{ii}}, \hspace{0.5cm} i = \overline{1, n}.
\end{equation}

Формула (7) определяет итерационный метод решения системы (6), называемый \\
методом Якоби или методом простой итерации.

Запишем этот метод для нашей системы:

\begin{equation}
    y^{k+1}_i = \frac{a_i}{a_i + a_{i+1} + h^2 g_i} y^k_{i-1} + \frac{a_{i + 1}}{a_i + a_{i+1} + h^2 g_i} y^k_{i+1} + \frac{f_i h^2}{a_i + a_{i+1} + h^2 g_i}, $$$$
    i = \overline{1, n - 1}, \hspace{0.5cm} y^0_i = 0, \hspace{0.5cm} y^k_0 = y^k_n = 0 \hspace{0.5cm} \forall k\\
\end{equation}

Вычисления продолжать до тех пор, пока не выполнится условие:

\begin{equation*}
  \max_{1 \le i \le n - 1} \left|r_{i}^{k}\right| \le \varepsilon,
\end{equation*}
где $r^k$ - вектор невязки для $k$-ой итерации $r^k = Ay^k - f$, \hspace{0.5cm} $\varepsilon = h^3$.
\newpage
\begin{enumerate}[label = \arabic*.]
  \item {Результаты метода Якоби для n = 10:
    \begin{table}[h]
      \centering
      \begin{tabular}{|c|c|c|c|c|}
        \hline
        $ih$ & $y_i$ & $y^k_i$ & $\left|y_i-y^k_i\right|$ & $k$\\ \hline
        0.1 & -0.000695158 & -0.000700758 &   5.5999e-06 & 145\\ \hline
        0.2 &  -0.00039375 & -0.000404145 &  1.03947e-05 & 145\\ \hline
        0.3 &   0.00300748 &   0.00299397 &  1.35191e-05 & 145\\ \hline
        0.4 &    0.0116375 &    0.0116219 &  1.55989e-05 & 145\\ \hline
        0.5 &    0.0264841 &    0.0264685 &  1.55772e-05 & 145\\ \hline
        0.6 &    0.0462174 &    0.0462028 &  1.46068e-05 & 145\\ \hline
        0.7 &    0.0660077 &    0.0659958 &  1.18485e-05 & 145\\ \hline
        0.8 &    0.0763388 &    0.0763303 &  8.51864e-06 & 145\\ \hline
        0.9 &    0.0618207 &    0.0618164 &  4.28485e-06 & 145\\ \hline
      \end{tabular}
      \caption*{\small{Таблица 3 - таблица значений для формулы метода Якоби при n = 10}}
    \end{table}
  }
  \item {Для n = 20:
    \begin{table}[h]
      \centering
      \begin{tabular}{|c|c|c|c|c|}
        \hline
        $ih$ & $y_i$ & $y^k_i$ & $\left|y_i-y^k_i\right|$ & $k$\\ \hline
        0.05 & -7.79134e-05 & -7.83545e-05 &  4.41101e-07 & 733\\ \hline
        0.1 &  -8.5795e-05 & -8.66476e-05 &  8.52527e-07 & 733\\ \hline
        0.15 &  0.000152985 &  0.000151761 &  1.22406e-06 & 733\\ \hline
        0.2 &  0.000889159 &  0.000887605 &  1.55335e-06 & 733\\ \hline
        0.25 &   0.00241095 &   0.00240912 &   1.8293e-06 & 733\\ \hline
        0.3 &   0.00500727 &   0.00500522 &  2.05459e-06 & 733\\ \hline
        0.35 &   0.00893085 &   0.00892863 &  2.21834e-06 & 733\\ \hline
        0.4 &    0.0143611 &    0.0143588 &  2.32761e-06 & 733\\ \hline
        0.45 &    0.0213673 &    0.0213649 &  2.37278e-06 & 733\\ \hline
        0.5 &    0.0298709 &    0.0298686 &  2.36437e-06 & 733\\ \hline
        0.55 &     0.039609 &    0.0396067 &  2.29488e-06 & 733\\ \hline
        0.6 &    0.0500966 &    0.0500944 &  2.17716e-06 & 733\\ \hline
        0.65 &    0.0605893 &    0.0605873 &  2.00646e-06 & 733\\ \hline
        0.7 &    0.0700467 &    0.0700449 &  1.79667e-06 & 733\\ \hline
        0.75 &    0.0770939 &    0.0770924 &  1.54618e-06 & 733\\ \hline
        0.8 &    0.0799855 &    0.0799842 &  1.26863e-06 & 733\\ \hline
        0.85 &    0.0765672 &    0.0765662 &  9.65576e-07 & 733\\ \hline
        0.9 &     0.064239 &    0.0642384 &  6.49247e-07 & 733\\ \hline
        0.95 &    0.0399178 &    0.0399175 &  3.24129e-07 & 733\\ \hline
      \end{tabular}
      \caption*{\small{Таблица 4 - таблица значений для формулы метода Якоби при n = 20}}
    \end{table}
  }
\end{enumerate}
\newpage



\subsection{Метод верхней релаксации}

\hspace{1cm} Во многих ситуациях существенного ускорения сходимости можно добиться
за счет введения так называемого итерационного параметра. Рассмотрим итерационнный процесс:
\begin{equation}
  x^{k+1}_i = (1 - \omega)x^k_i + \omega(-\sum\limits_{j=1}^{i-1}\frac{a_{ij}}{a_{ii}} x^{k + 1}_j - \sum\limits_{j = i+1}^{n}\frac{a_{ij}}{a_{ii}} x^k_j + \frac{b_i}{a_{ii}}), $$$$
  i = \overline{1, n}, \hspace{0.5cm} k = 0, 1, \dots
\end{equation}

Этот метод называется методом релаксации — одним из наиболее эффективных и 
широко используемых итерационных методов для решения систем линейных алгебраических уравнений. 
Значение $\omega$ - называется релаксационным параметром. При $\omega$ = 1 метод переходит в 
метод Зейделя. При $\omega\in{(1,2)}$ - это метод верхней релаксации, при $\omega\in{(0,1)}$ - метод нижней релаксации. 
Ясно, что по затратам памяти и объему вычислений на каждом шаге итераций метод релаксации 
не отличается от метода Зейделя. Мы исследуем сходимость метода релаксации 
в случае, когда матрица А симметрична и положительно определена. С этой целью перепишем его
в матричном виде. Обозначим через L нижнюю треугольную матрицу с
нулевой главной диагональю; элементы, стоящие под главной диагональю
матрицы $L$, cоответствующими элементами матрицы $А$. Через $D$, обозначим диагональную матрицу, 
на диагонали которой стоят диагональные элементы матрицы А. Понятно, что $A = L + D + L^T$. 
Нетрудно убедиться, что равенство (9) с учетом введенных обозначений принимает вид:
\begin{equation*}
  Dx^{k + 1} = (1 - \omega)Dx^k + \omega(-Lx^{k + 1} - L^T x^k + b). 
\end{equation*}
\hspace{1cm} После элементарных преобразований получим, что
\begin{equation}
  B\frac{x^{k + 1} - x^k}{\omega} + Ax^k = b, \hspace{0.5cm}\text{где} \hspace{0.2cm}B = D + \omega L
\end{equation}


Естественно параметр $\omega$ следует выбирать так, чтобы метод релаксации 
сходился наиболее быстро. В нашем случае выберем $\omega \in{(1, 2)}$ и заполним таблицу, в которой 
первая строка - это релаксационный параметр, вторая - кол-во итераций, потребовавшихся для достижения 
заданной точности.
\newpage

\begin{enumerate}[label = \arabic*.]
  \item {Результат метода верхней релаксации при n = 10:
      \begin{table}[h]
        \centering
        \begin{tabular}{|c|c|c|c|c|c|c|c|c|c|}
          \hline
          $w$ & 1.1 & 1.2 & 1.3 & 1.4 & 1.5 & 1.6 & 1.7 & 1.8 & 1.9\\ \hline
          $k$ & 29 & 28 & 25 & 21 & 16 & 20 & 28 & 39 & 74\\ \hline
        \end{tabular}
        \caption*{\small{Таблица 7 - таблица значений для формулы метода релаксации при n = 10}}
      \end{table}

  \hspace{0.5cm} При n = 10 погрешность меньше всего при $\omega$ = 1.5. 
  Сравним значения при данном $\omega$ с методом прогонки и построим таблицу:
  \begin{table}[h]
    \centering
    \begin{tabular}{|c|c|c|c|c|}
      \hline
      $ih$ & $y_i$ & $y^k_i$ & $\left|y_i-y^k_i\right|$ & $k$\\ \hline
      0.1 & -0.000695158 &  -0.00869934 &   0.00800418 & 16\\ \hline
      0.2 &  -0.00039375 &   -0.0158154 &    0.0154216 & 16\\ \hline
      0.3 &   0.00300748 &   -0.0193951 &    0.0224026 & 16\\ \hline
      0.4 &    0.0116375 &   -0.0174571 &    0.0290946 & 16\\ \hline
      0.5 &    0.0264841 &  -0.00912885 &    0.0356129 & 16\\ \hline
      0.6 &    0.0462174 &   0.00415891 &    0.0420585 & 16\\ \hline
      0.7 &    0.0660077 &    0.0174865 &    0.0485212 & 16\\ \hline
      0.8 &    0.0763388 &     0.021256 &    0.0550829 & 16\\ \hline
      0.9 &    0.0618207 &            0 &    0.0618207 & 16\\ \hline
    \end{tabular}
    \caption*{\small{Таблица 8 - таблица значений метода верхней релаксации при $\omega$ = 1.5 и n = 10}}
  \end{table}
  }
  \item {Повторим дейстия при n = 20:
    \begin{table}[h]
      \centering
      \begin{tabular}{|c|c|c|c|c|c|c|c|c|c|c|}
        \hline
        $w$ & 1.05 & 1.15 & 1.25 & 1.35 & 1.45 & 1.55 & 1.65 & 1.75 & 1.85 & 1.95 \\ \hline
        $k$ & 254 & 214 & 179 & 148 & 120 & 93 & 66 & 46 & 71 & 218 \\ \hline
      \end{tabular}
      \caption*{\small{Таблица 9 - таблица значений для формулы метода Релаксации при n = 20}}
    \end{table}
  \newpage
  \hspace{0.5cm} При n = 20 погрешность меньше всего при $\omega$ = 1.75. 
  Сравним значения при данном $\omega$ с методом прогонки:
      \begin{table}[h]
        \centering
        \begin{tabular}{|c|c|c|c|c|}
          \hline
          $ih$ & $y_i$ & $y^k_i$ & $\left|y_i-y^k_i\right|$ & $k$\\ \hline
          0.05 & -7.79134e-05 &  -0.00258756 &   0.00250965 & 46\\ \hline
          0.1 &  -8.5795e-05 &   -0.0049969 &   0.00491111 & 46\\ \hline
          0.15 &  0.000152985 &  -0.00706679 &   0.00721977 & 46\\ \hline
          0.2 &  0.000889159 &  -0.00856025 &   0.00944941 & 46\\ \hline
          0.25 &   0.00241095 &  -0.00920148 &    0.0116124 & 46\\ \hline
          0.3 &   0.00500727 &  -0.00871282 &    0.0137201 & 46\\ \hline
          0.35 &   0.00893085 &  -0.00685186 &    0.0157827 & 46\\ \hline
          0.4 &    0.0143611 &  -0.00344859 &    0.0178097 & 46\\ \hline
          0.45 &    0.0213673 &   0.00155737 &    0.0198099 & 46\\ \hline
          0.5 &    0.0298709 &   0.00807957 &    0.0217914 & 46\\ \hline
          0.55 &     0.039609 &    0.0158474 &    0.0237616 & 46\\ \hline
          0.6 &    0.0500966 &    0.0243689 &    0.0257276 & 46\\ \hline
          0.65 &    0.0605893 &    0.0328928 &    0.0276966 & 46\\ \hline
          0.7 &    0.0700467 &    0.0403719 &    0.0296748 & 46\\ \hline
          0.75 &    0.0770939 &    0.0454254 &    0.0316686 & 46\\ \hline
          0.8 &    0.0799855 &    0.0463016 &    0.0336839 & 46\\ \hline
          0.85 &    0.0765672 &    0.0408404 &    0.0357268 & 46\\ \hline
          0.9 &     0.064239 &    0.0264361 &    0.0378029 & 46\\ \hline
          0.95 &    0.0399178 &            0 &    0.0399178 & 46\\ \hline
        \end{tabular}
        \caption*{\small{Таблица 10 - таблица значений метода верхней релаксации при $\omega$ = 1.75 и n = 20}}
      \end{table}
  }
\end{enumerate}
\newpage



\subsection{Метод наискорейшего спуска}

\hspace{1cm} Опишем метод минимизации функционала. Будем двигаться из точки начального 
приближения $x^0$ в направлении наибыстрейшего убывания функционала $F$, 
то есть следующее приближение будем разыскивать так:
$x^1 = x^0 - \tau gradF(x^0).$ Формула:
\begin{equation}
  x^{k+1}_i = -\sum\limits_{j = 1}^{i - 1}\frac{a_{ij}}{a_{ii}} x^{k + 1}_j - \sum\limits_{j = i + 1}^{n}\frac{a_{ij}}{a_{ii}} x^k_j + \frac{b_i}{a_{ii}},
\end{equation}
показывает, что $gradF(x^k) = 2(Ax^0 - b)$. Вектор $r_0 = Ax^0 - b$ принято называть невязкой.
Для сокращения записей удобно обозначить  $2\tau$ вновь через $\tau$. Таким образом, 
$x^1 = x^0 - \tau r^0$.

Параметр $\tau$ выеберем так, чтобы значение $F(x^1)$ было минимальным. 
Получим $F(x^1) = F(x^0 - \tau r^0) = F(x^0) - 2\tau(r^0, r^0) + \tau^2(Ar^0, r^0)$, следовательно,
минимум $F(x^1)$ достигается при $\tau = \tau_* = \frac{(r^0, r^0)}{(Ar^0, r^0)}$.

Таким образом, мы пришли к следующему итерационному методу:
\begin{equation}
  x^{k + 1} = x^k - \tau_* r^k, \hspace{0.2cm} r^k = Ax^k - b, \hspace{0.2cm} \tau_* = \frac{(r^k, r^k)}{(Ar^k, r^k)}, k = 0, 1, \dots
\end{equation}

Метод (12) называют методом наискорейшего спуска. По сравнению с методом простой итерации этот метод
требует на каждом шаге итераций проведения дополнительной работы по вычислению параметра $\tau_*$. 
Вследствие этого происходит адаптация к оптимальной скорости сходимости.

Первый столбец таблицы метода наискорейшего спуска - это номер итерации, \\
умноженный на число узлов разбиения, второй - значения, полученные методом 
прогонки. Третий - значения, полученные методом наискорейшего спуска. В четвертом столбце находится значение погршености.
В пятом столбце - кол-во итераций, потребовавшихся для достижения заданной точности.

\begin{enumerate}[label = \arabic*.]
  \item{Результат метода наискорейшего спуска при n = 10:
    \begin{table}[h]
      \centering
      \begin{tabular}{|c|c|c|c|c|c|}
        \hline
        $ih$ & $y_i$ & $y^k_i$ & $\left|y_i-y^k_i\right|$ & $k$\\ \hline
        0.1 & -0.000695158 &  -0.00724094 &   0.00654579 & 13\\ \hline
        0.2 &  -0.00039375 &   -0.0133147 &     0.012921 & 13\\ \hline
        0.3 &   0.00300748 &   -0.0163127 &    0.0193201 & 13\\ \hline
        0.4 &    0.0116375 &   -0.0144059 &    0.0260434 & 13\\ \hline
        0.5 &    0.0264841 &  -0.00620665 &    0.0326907 & 13\\ \hline
        0.6 &    0.0462174 &   0.00622007 &    0.0399974 & 13\\ \hline
        0.7 &    0.0660077 &    0.0192136 &    0.0467941 & 13\\ \hline
        0.8 &    0.0763388 &    0.0218845 &    0.0544543 & 13\\ \hline
        0.9 &    0.0618207 &            0 &    0.0618207 & 13\\ \hline
      \end{tabular}
      \caption*{\small{Таблица 11 - таблица значений для формулы метода наискорейшего спуска при n = 10}}
    \end{table}
  }
  \newpage
  \item{При n = 20
    \begin{table}[h]
      \centering
      \begin{tabular}{|c|c|c|c|c|}
      \hline
      $ih$ & $y_i$ & $y^k_i$ & $\left|y_i-y^k_i\right|$ & $k$\\ \hline
      0.05 & -7.79134e-05 &    -0.003279 &   0.00320109 & 75\\ \hline
      0.1 &  -8.5795e-05 &  -0.00634352 &   0.00625773 & 75\\ \hline
      0.15 &  0.000152985 &  -0.00902848 &   0.00918146 & 75\\ \hline
      0.2 &  0.000889159 &   -0.0110897 &    0.0119789 & 75\\ \hline
      0.25 &   0.00241095 &   -0.0122406 &    0.0146515 & 75\\ \hline
      0.3 &   0.00500727 &   -0.0121903 &    0.0171976 & 75\\ \hline
      0.35 &   0.00893085 &   -0.0106814 &    0.0196123 & 75\\ \hline
      0.4 &    0.0143611 &  -0.00753157 &    0.0218927 & 75\\ \hline
      0.45 &    0.0213673 &  -0.00266403 &    0.0240313 & 75\\ \hline
      0.5 &    0.0298709 &    0.0038321 &    0.0260388 & 75\\ \hline
      0.55 &     0.039609 &     0.011714 &     0.027895 & 75\\ \hline
      0.6 &    0.0500966 &    0.0204382 &    0.0296584 & 75\\ \hline
      0.65 &    0.0605893 &    0.0293412 &    0.0312481 & 75\\ \hline
      0.7 &    0.0700467 &    0.0372015 &    0.0328451 & 75\\ \hline
      0.75 &    0.0770939 &    0.0428759 &     0.034218 & 75\\ \hline
      0.8 &    0.0799855 &    0.0442423 &    0.0357432 & 75\\ \hline
      0.85 &    0.0765672 &    0.0395435 &    0.0370237 & 75\\ \hline
      0.9 &     0.064239 &    0.0257245 &    0.0385145 & 75\\ \hline
      0.95 &    0.0399178 &            0 &    0.0399178 & 75\\ \hline
      \end{tabular}
      \caption*{\small{Таблица 12 - таблица значений для формулы метода наискорейшего спуска при n = 20}}
    \end{table}
  }
\end{enumerate}
\newpage

\begin{center}
\section{Выводы}
\end{center}

В процессе выполнения данной работы были получены знания решения систем 
линейных алгебраических уравнений методом прогонки и итерационными методами: 
Якоби, верхней релаксации, наискорейшего спуска. Исходя из этого мы сделали вывод, 
что решить систему линейных алгебраических уравнений методом верхней релаксации является наиболее эффективным
из всех методов, которые мы рассмотрели, так как за наименьшее количество разбиений мы получили более точный результат.
\newpage

\begin{center}
\section{Cписок литературы}
\end{center}
\begin{enumerate}
    \item Глазырина Л.Л., Карчевский М.М. Численные методы: учебное пособие. — Казань: Казан.
    ун-т, 2012. — 122 
    \item Глазырина Л.Л.. Практикум по курсу «Численные методы». Решение
    систем линейных уравнений: учеб. пособие. — Казань: Изд-во Казан. ун-та, 2017. — 52 с.
\end{enumerate}
\newpage

\begin{center}
\section{Листинг программы}
\end{center}
\lstinputlisting[language=C++]{../src/methods.hpp}
\end{document}