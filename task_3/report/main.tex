\documentclass[a4paper,12pt]{article}
\usepackage[utf8x]{inputenc}
\usepackage[T2A]{fontenc}
\usepackage[russian,english]{babel}
\usepackage{amsmath}
\usepackage{cmap}
\usepackage{booktabs}
\usepackage{caption}
\usepackage{enumitem}
\usepackage{listings}
\usepackage{xcolor}
\usepackage{setspace}
\usepackage[left=3cm, right=1.5cm, top=2cm, bottom=2cm]{geometry}
\renewcommand{\labelenumii}{\arabic{enumi}.\arabic{enumii}.}
\lstset{
    language=C++,
    basicstyle=\small\ttfamily,
    keywordstyle=\color{blue},
    commentstyle=\color{green!40!black},
    stringstyle=\color{purple},
    numbers=left,
    numberstyle=\tiny,
    numbersep=5pt,
    breaklines=true,
    frame=single,
    backgroundcolor=\color{gray!10},
    rulecolor=\color{black!30},
    showstringspaces=false,
    extendedchars=\true, % Включение расширенных символов, включая русский текст
}
\begin{document}

\begin{center}
\hfill \break
\textbf{\large{Министерство науки и высшего образования Российской Федерации\\
Федеральное государственное автономное образовательное\\
учреждение высшего образования}}
\\
\large{\textbf{«КАЗАНСКИЙ (ПРИВОЛЖСКИЙ) ФЕДЕРАЛЬНЫЙ УНИВЕРСИТЕТ»}}\\
\hfill \break
\large{ИНСТИТУТ ВЫЧИСЛИТЕЛЬНОЙ МАТЕМАТИКИ\\ И ИНФОРМАЦИОННЫХ ТЕХНОЛОГИЙ}\\
 \hfill \break
\large{Кафедра прикладной математики и искусственого интеллекта}\\
\hfill\break
\hfill \break
\large{Направление подготовки: 01.03.04 – Прикладная математика}\\
\hfill \break
\hfill \break
\textbf{\large{ОТЧЁТ}}\\
\large{По дисциплине <<Численные методы>>}\\
\large{на тему:}\\
\large{<<Система линейных алгебраических уравнений>>}\\
\hfill \break
\hfill \break
\end{center}

\hfill \break
\begin{flushright}
			
    \large{Выполнил:}
    
    \large{студент группы 09-221}
    
    \large{Саитов М.А.}
    
    \large{Преподаватель:}
    
    \large{Глазырина О.В.}
    
\end{flushright}
\vfill
\begin{center} \large{Казань, 2024 год} \end{center}
\thispagestyle{empty}
 

\newpage
\begin{center}
\renewcommand{\contentsname}{Содержание}
\fontsize{14}{1.15}\selectfont
\mdseries\selectfont{\tableofcontents}
\newpage
\end{center}
\setlength{\parindent}{1.25cm}
\newpage
\selectfont\onehalfspacing{

\section{Постановка задачи}
\newpage

\section{Ход работы}
\begin{enumerate}[label = \arabic*.]
    \item {Метод прогонки:
        \begin{table}[h]
        \centering
        \begin{tabular}{|c|c|c|c|}
            \hline
            $i*h$ & $y_i$ & $u(ih)$ & $\left|y_i-u(ih)\right|$\\
            \hline
            0.1 & -0.000695158 &        9e-05 &  0.000785158\\
            \hline
            0.2 &  -0.00039375 &      0.00128 &   0.00167375\\
            \hline
            0.3 &   0.00300748 &      0.00567 &   0.00266252\\
            \hline
            0.4 &    0.0116375 &      0.01536 &   0.00372252\\
            \hline
            0.5 &    0.0264841 &      0.03125 &   0.00476594\\
            \hline
            0.6 &    0.0462174 &      0.05184 &   0.00562257\\
            \hline
            0.7 &    0.0660077 &      0.07203 &   0.00602232\\
            \hline
            0.8 &    0.0763388 &      0.08192 &   0.00558118\\
            \hline
            0.9 &    0.0618207 &      0.06561 &   0.00378929\\
            \hline
        \end{tabular}
        \caption*{\small{Таблица 1 - таблица значений для формул метода прогонки при n = 10}}
        \end{table}
    }
    \begin{table}[h]
    \centering
    \begin{tabular}{|c|c|c|c|}
        \hline
        $i*h$ & $y_i$ & $u(ih)$ & $\left|y_i-u(ih)\right|$\\
        \hline
        0.05 & -7.79134e-05 &   5.9375e-06 &  8.38509e-05\\
        \hline
        0.1 &  -8.5795e-05 &        9e-05 &  0.000175795\\
        \hline
        0.15 &  0.000152985 &  0.000430313 &  0.000277328\\
        \hline
        0.2 &  0.000889159 &      0.00128 &  0.000390841\\
        \hline
        0.25 &   0.00241095 &   0.00292969 &   0.00051874\\
        \hline
        0.3 &   0.00500727 &      0.00567 &  0.000662726\\
        \hline
        0.35 &   0.00893085 &   0.00975406 &  0.000823217\\
        \hline
        0.4 &    0.0143611 &      0.01536 &  0.000998868\\
        \hline
        0.45 &    0.0213673 &    0.0225534 &   0.00118616\\
        \hline
        0.5 &    0.0298709 &      0.03125 &   0.00137908\\
        \hline
        0.55 &     0.039609 &    0.0411778 &    0.0015688\\
        \hline
        0.6 &    0.0500966 &      0.05184 &   0.00174344\\
        \hline
        0.65 &    0.0605893 &    0.0624772 &   0.00188784\\
        \hline
        0.7 &    0.0700467 &      0.07203 &   0.00198335\\
        \hline
        0.75 &    0.0770939 &    0.0791016 &   0.00200763\\
        \hline
        0.8 &    0.0799855 &      0.08192 &    0.0019345\\
        \hline
        0.85 &    0.0765672 &    0.0783009 &   0.00173375\\
        \hline
        0.9 &     0.064239 &      0.06561 &     0.001371\\
        \hline
        0.95 &    0.0399178 &    0.0407253 &  0.000807495\\
        \hline
    \end{tabular}
    \caption*{\small{Таблица 2 - таблица значений для формул метода прогонки при n = 20}}
    \end{table}
    \newpage
    \item {Метод Якоби:
        \begin{table}[h]
          \centering
          \begin{tabular}{|c|c|c|c|}
            \hline
            $i*h$ & $y_i$ & $u(ih)$ & $\left|y_i-u(ih)\right|$\\
            \hline
            0.1 &     0.102819 &        9e-05 &     0.102729\\
            \hline
            0.2 &     0.199156 &      0.00128 &     0.197876\\
            \hline
            0.3 &     0.293056 &      0.00567 &     0.287386\\
            \hline
            0.4 &     0.388895 &      0.01536 &     0.373535\\
            \hline
            0.5 &     0.488671 &      0.03125 &     0.457421\\
            \hline
            0.6 &     0.592915 &      0.05184 &     0.541075\\
            \hline
            0.7 &     0.697403 &      0.07203 &     0.625373\\
            \hline
            0.8 &     0.794114 &      0.08192 &     0.712194\\
            \hline
            0.9 &     0.868238 &      0.06561 &     0.802628\\
            \hline
          \end{tabular}
          \caption*{\small{Таблица 3 - таблица значений для формулы метода Якоби при n = 10}}
        \end{table}
        \begin{table}[h]
            \centering
            \begin{tabular}{|c|c|c|c|}
                \hline
                $i*h$ & $y_i$ & $u(ih)$ & $\left|y_i-u(ih)\right|$\\
                \hline
                0.05 &    0.0562275 &   5.9375e-06 &    0.0562215\\
                \hline
                0.1 &     0.110114 &        9e-05 &     0.110024\\
                \hline
                0.15 &     0.162167 &  0.000430313 &     0.161737\\
                \hline
                0.2 &     0.212985 &      0.00128 &     0.211705\\
                \hline
                0.25 &     0.263096 &   0.00292969 &     0.260167\\
                \hline
                0.3 &     0.313096 &      0.00567 &     0.307426\\
                \hline
                0.35 &     0.363411 &   0.00975406 &     0.353657\\
                \hline
                0.4 &     0.414498 &      0.01536 &     0.399138\\
                \hline
                0.45 &     0.466555 &    0.0225534 &     0.444002\\
                \hline
                0.5 &     0.519757 &      0.03125 &     0.488507\\
                \hline
                0.55 &     0.573939 &    0.0411778 &     0.532761\\
                \hline
                0.6 &     0.628847 &      0.05184 &     0.577007\\
                \hline
                0.65 &     0.683819 &    0.0624772 &     0.621342\\
                \hline
                0.7 &     0.738021 &      0.07203 &     0.665991\\
                \hline
                0.75 &     0.790159 &    0.0791016 &     0.711058\\
                \hline
                0.8 &      0.83867 &      0.08192 &      0.75675\\
                \hline
                0.85 &     0.881484 &    0.0783009 &     0.803183\\
                \hline
                0.9 &     0.916161 &      0.06561 &     0.850551\\
                \hline
                0.95 &     0.939716 &    0.0407253 &     0.898991\\
                \hline
            \end{tabular}
            \caption*{\small{Таблица 4 - таблица значений для формулы метода Якоби при n = 20}}
          \end{table}
    }
    \newpage
    \item {Метод Зейделя:
        \begin{table}[h]
          \centering
          \begin{tabular}{|c|c|c|c|c|}
            \hline
            $i*h$ & $y_i$ & $u(ih)$ & $\left|y_i-u(ih)\right|$\\
            \hline
            0.1 &     0.103011 &        9e-05 &     0.102921\\
            \hline
            0.2 &     0.199491 &      0.00128 &     0.198211\\
            \hline
            0.3 &     0.293725 &      0.00567 &     0.288055\\
            \hline
            0.4 &     0.389619 &      0.01536 &     0.374259\\
            \hline
            0.5 &     0.489654 &      0.03125 &     0.458404\\
            \hline
            0.6 &     0.593779 &      0.05184 &     0.541939\\
            \hline
            0.7 &     0.698296 &      0.07203 &     0.626266\\
            \hline
            0.8 &     0.794716 &      0.08192 &     0.712796\\
            \hline
            0.9 &     0.868608 &      0.06561 &     0.802998\\
            \hline
          \end{tabular}
          \caption*{\small{Таблица 5 - таблица значений для формулы метода Зейделя при n = 10}}
        \end{table}
        \begin{table}[h]
          \centering
          \begin{tabular}{|c|c|c|c|c|}
            \hline
            $i*h$ & $y_i$ & $u(ih)$ & $\left|y_i-u(ih)\right|$\\
            \hline
            0.05 &    0.0562405 &   5.9375e-06 &    0.0562346\\
            \hline
            0.1 &     0.110139 &        9e-05 &     0.110049\\
            \hline
            0.15 &     0.162222 &  0.000430313 &     0.161792\\
            \hline
            0.2 &     0.213055 &      0.00128 &     0.211775\\
            \hline
            0.25 &     0.263206 &   0.00292969 &     0.260276\\
            \hline
            0.3 &     0.313218 &      0.00567 &     0.307548\\
            \hline
            0.35 &     0.363576 &   0.00975406 &     0.353822\\
            \hline
            0.4 &      0.41467 &      0.01536 &      0.39931\\
            \hline
            0.45 &     0.466766 &    0.0225534 &     0.444212\\
            \hline
            0.5 &     0.519965 &      0.03125 &     0.488715\\
            \hline
            0.55 &     0.574174 &    0.0411778 &     0.532996\\
            \hline
            0.6 &     0.629068 &      0.05184 &     0.577228\\
            \hline
            0.65 &     0.684051 &    0.0624772 &     0.621574\\
            \hline
            0.7 &     0.738227 &      0.07203 &     0.666197\\
            \hline
            0.75 &     0.790358 &    0.0791016 &     0.711257\\
            \hline
            0.8 &     0.838832 &      0.08192 &     0.756912\\
            \hline
            0.85 &     0.881621 &    0.0783009 &      0.80332\\
            \hline
            0.9 &     0.916252 &      0.06561 &     0.850642\\
            \hline
            0.95 &     0.939766 &    0.0407253 &     0.899041\\
            \hline
            \end{tabular}
            \caption*{\small{Таблица 6 - таблица значений для формулы метода Зейделя при n = 20}}
          \end{table}
    }
    \newpage
    \item {Метод Релаксации:
        \begin{table}[h]
          \centering
          \begin{tabular}{|c|c|}
            \hline
            $w$ & $k$\\
            \hline
            0.1 & 497\\
            \hline
            0.2 & 58\\
            \hline
            0.3 & 24\\
            \hline
            0.4 & 13\\
            \hline
            0.5 & 8\\
            \hline
            0.6 & 6\\
            \hline
            0.7 & 4\\
            \hline
            0.8 & 3\\
            \hline
            0.9 & 3\\
            \hline
            1 & 2\\
            \hline
          \end{tabular}
          \caption*{\small{Таблица 7 - таблица значений для формулы метода Релаксации при n = 10}}
        \end{table}
        \begin{table}[h]
          \centering
          \begin{tabular}{|c|c|}
            \hline
            $w$ & $k$\\
            \hline
            0.1 & 2446\\
            \hline
            0.2 & 243\\
            \hline
            0.3 & 98\\
            \hline
            0.4 & 52\\
            \hline
            0.5 & 33\\
            \hline
            0.6 & 23\\
            \hline
            0.7 & 16\\
            \hline
            0.8 & 12\\
            \hline
            0.9 & 10\\
            \hline
            1 & 8\\
            \hline
            \end{tabular}
            \caption*{\small{Таблица 8 - таблица значений для формулы метода Релаксации при n = 20}}
          \end{table}
    }
    \newpage
    \item {Метод наискорейшего спуска:
        \begin{table}[h]
          \centering
          \begin{tabular}{|c|c|c|c|c|}
            \hline
            $i*h$ & $y_i$ & $u(ih)$ & $\left|y_i-u(ih)\right|$\\
            \hline
            0.1 &     0.102819 &        9e-05 &     0.102729\\
            \hline
            0.2 &     0.199156 &      0.00128 &     0.197876\\
            \hline
            0.3 &     0.293056 &      0.00567 &     0.287386\\
            \hline
            0.4 &     0.388895 &      0.01536 &     0.373535\\
            \hline
            0.5 &     0.488671 &      0.03125 &     0.457421\\
            \hline
            0.6 &     0.592915 &      0.05184 &     0.541075\\
            \hline
            0.7 &     0.697403 &      0.07203 &     0.625373\\
            \hline
            0.8 &     0.794114 &      0.08192 &     0.712194\\
            \hline
            0.9 &     0.868238 &      0.06561 &     0.802628\\
            \hline
          \end{tabular}
          \caption*{\small{Таблица 9 - таблица значений для формулы метода наискорейшего спуска при n = 10}}
        \end{table}
        \begin{table}[h]
          \centering
          \begin{tabular}{|c|c|c|c|c|}
            \hline
            $i*h$ & $y_i$ & $u(ih)$ & $\left|y_i-u(ih)\right|$\\
            \hline
            0.05 &    0.0562275 &   5.9375e-06 &    0.0562215\\
            \hline
            0.1 &     0.110114 &        9e-05 &     0.110024\\
            \hline
            0.15 &     0.162167 &  0.000430313 &     0.161737\\
            \hline
            0.2 &     0.212985 &      0.00128 &     0.211705\\
            \hline
            0.25 &     0.263096 &   0.00292969 &     0.260167\\
            \hline
            0.3 &     0.313096 &      0.00567 &     0.307426\\
            \hline
            0.35 &     0.363411 &   0.00975406 &     0.353657\\
            \hline
            0.4 &     0.414498 &      0.01536 &     0.399138\\
            \hline
            0.45 &     0.466555 &    0.0225534 &     0.444002\\
            \hline
            0.5 &     0.519757 &      0.03125 &     0.488507\\
            \hline
            0.55 &     0.573939 &    0.0411778 &     0.532761\\
            \hline
            0.6 &     0.628847 &      0.05184 &     0.577007\\
            \hline
            0.65 &     0.683819 &    0.0624772 &     0.621342\\
            \hline
            0.7 &     0.738021 &      0.07203 &     0.665991\\
            \hline
            0.75 &     0.790159 &    0.0791016 &     0.711058\\
            \hline
            0.8 &      0.83867 &      0.08192 &      0.75675\\
            \hline
            0.85 &     0.881484 &    0.0783009 &     0.803183\\
            \hline
            0.9 &     0.916161 &      0.06561 &     0.850551\\
            \hline
            0.95 &     0.939716 &    0.0407253 &     0.898991\\
            \hline
            \end{tabular}
            \caption*{\small{Таблица 10 - таблица значений для формулы метода наискорейшего спуска при n = 20}}
          \end{table}
    }
    \newpage
\end{enumerate}

\section{Выводы}
\hspace{1.25cm}Проделав все вычисления, можно сделать выводы, что более комплексные методы вычисления интеграла, как формула Гаусса и Симпсона, показыают наилучшие результа\-ты за меньшее количество разбиений. В это же время худшие результаты вычисления показыают методы правых прямоугольников и метод трапеций, приводя к довольно большому значению ошибки.}
\clearpage
\section{Листинг программы}
\lstinputlisting[language=C++]{../src/methods.hpp}
\end{document}